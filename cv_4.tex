%%%%%%%%%%%%%%%%%%%%%%%%%%%%%%%%%%%%%%%%%
% Medium Length Professional CV
% LaTeX Template
% Version 2.0 (8/5/13)
%
% This template has been downloaded from:
% http://www.LaTeXTemplates.com
%
% Original author:
% Trey Hunner (http://www.treyhunner.com/)
%
% Important note:
% This template requires the resume.cls file to be in the same directory as the
% .tex file. The resume.cls file provides the resume style used for structuring the
% document.
%
%%%%%%%%%%%%%%%%%%%%%%%%%%%%%%%%%%%%%%%%%

%----------------------------------------------------------------------------------------
%	PACKAGES AND OTHER DOCUMENT CONFIGURATIONS
%----------------------------------------------------------------------------------------

\documentclass{resume} % Use the custom resume.cls style

\usepackage[left=0.75in,top=0.6in,right=0.75in,bottom=0.6in]{geometry} % Document margins
\usepackage{enumitem}
\usepackage{etaremune}
\usepackage{url}
%\usepackage{xeCJK}
\usepackage{lettrine}
\usepackage[UTF8]{ctex}

\name{Fanglin Sun} % Your name
\address{320 Donggang West Road \\ Lanzhou, Gansu 730000 China} % Your address
%\address{123 Pleasant Lane \\ City, State 12345} % Your secondary addess (optional)
\address{+86~139~1982~1909 \\ flsun@lzb.ac.cn} % Your phone number and email

\begin{document}
%----------------------------------------------------------------------------------------
%	quote SECTION
%----------------------------------------------------------------------------------------

\begin{rSection}{}
\smallskip
\smallskip
\smallskip
\centering{
{\it "Mountain meteorology, Monsoon, The Himalayas, and Boundary-layer processes."}
}
\smallskip
\end{rSection}


%----------------------------------------------------------------------------------------
%	EDUCATION SECTION
%----------------------------------------------------------------------------------------

\begin{rSection}{Education}

{\bf PhD, Atmospheric Physical and Environmental Science} (combined MS/PhD program)\\
%\ dff \ \hfill {\em September 2008 -- June 2002}
\rightline{ {\em September 2002 -- June 2004 \ \ \ }} \\ 
Cold and Arid Regions Environmental and Engineering Research Institute (CAREERI), CAS, Lanzhou, China
\begin{enumerate}[leftmargin=!,labelindent=5pt,itemindent=-15pt]
\item[] Dissertation: Structure and Evolution over Atmospheric Boundary Layer in Mount Everest Region; Supervisor: Yaoming Ma (ITP, CAS)
\end{enumerate}
{\bf Bachelor, Atmospheric Physics and Sounding} \hfill {\em September 1998 -- June 2002}\\
Nanjing Institute of Meteorology, Nanjing, China
\begin{enumerate}[leftmargin=!,labelindent=5pt,itemindent=-15pt]
\item[] Meteorology, Atmospheric Dynamics, Satellite and Radar Meteorology, Climatology,\\
Electronic Engineering
\end{enumerate}
\end{rSection}

%----------------------------------------------------------------------------------------
%	WORK EXPERIENCE SECTION
%----------------------------------------------------------------------------------------

\begin{rSection}{Experience}

\begin{rSubsection}{Assistant Professor}{September 2008 - Present}{Cold and Arid Regions Environmental and Engineering Research Institute,CAS }{Lanzhou, China}
\item {\it Grant}: Natural Science foundation of China (NSFC), Investigation on the trans-Himalayas air flow in Arun Valley east to Mt.Everest, In-situ observation and numerical simulation. 2015 - 2018
\item {\it Grant}: NSFC, Local circulation and ABL structure in northern Mt.Qomolangma region. 2012-2014
\end{rSubsection}
\begin{rSubsection}{Participation in Field Campaigns}{2008 - 2010}{Technical and Scientific Advisor}{Central Tibet, Northwest China}
\item Participated in CAMP-Tibet, responsible for GPS radio sounding and EC observation on central Tibetan Plateau. 
\item Participated in Heihe Integrated Remote Sensing Joint Experiment, Support for installation of Eddy Covariance System, Automatic weather station and GPS radio sounding in Black River region, northwest China.
\end{rSubsection}
\begin{rSubsection}{Internationl communication}{}{}{}
\item University of Bayreuth, Germany. November, 2010. Tibetan Plateau EC data processing using TK2 software. 
\item Kyoto University, Japan. December, 2009. WRF applicaiton on water and energy exchange on central Tibetan Plateau.
\end{rSubsection}
\end{rSection}

%----------------------------------------------------------------------------------------
%	TECHNICAL PUBLICATIONS SECTION
%----------------------------------------------------------------------------------------

\begin{rSection}{PUBLICATIONS}
\begin{etaremune}
  %\item {\bf Sun, FL}, Ma YM, Hu ZY, 2018. Local meteorology in a northern Himalayan valley near Mount Everest and its response to seasonal transitions. Sciences in Cold and Arid Regions, 10(6): 1-9
  \item {\bf Sun, F}, Y.~Ma, Z.~Hu, M.~Li, G.~Tartari, F.~Salerno, T.~Gerken, P.~Bonasoni, P.~Cristofanelli, and E.~Vuillermoz, 2018: Mechanism of Daytime Strong Winds on the Northern Slopes of Himalayas, near Mount Everest: Observation and Simulation. Journal of Applied Meteorology and Climatology, 57, 255-272, \url{https://doi.org/10.1175/JAMC-D-16-0409.1}
  \item {\bf Sun, F}; Ma, Y; Hu, Z; Li, M; Gerken, T; Zhang, L; Han, C; Sun, G. 2017. Observation of strong winds on northern slopes of Mount Everest in the monsoon season, Arctic, Antarctic, and Alpine Research, Vol. 49, No. 4, pp. 687-697. https://doi.org/10.1657/AAAR0016-010
  \item {\bf Sun, F.~}, Y.~Ma, M.~Li, W.~Ma, H.~Tian, and S.~Metzger,2007, Boundary layer effects above a Himalayan valley near Mount Everest, Geophysical Research Letters, 34, L08808, \url{https://doi.org/10.1029/ 2007GL029484}
  \item Ma, Y., Hu, Z., Xie, Z., Ma, W., Wang, B., Chen, X., Li, M., Zhong, L., {\bf Sun, F.}, Gu, L., Han, C., Zhang, L., Liu, X., Ding, Z., Sun, G., Wang, S., Wang, Y., and Wang, Z. 2020. A long-term (2005–2016) dataset of hourly integrated land–atmosphere interaction observations on the Tibetan Plateau, Earth Syst. Sci. Data, 12, 2937–2957, \url{https://doi.org/10.5194/essd-12-2937-2020}
  \item Sun, G.; Hu, Z.; Wang, J.; Ma, W.; Gu, L.; {\bf Sun, F.}; Xie, Z.; Yan, X. The spatial heterogeneity of land surface conditions and its influence on surface fluxes over a typical underlying surface in the tibetan plateau. Theor. Appl. Climatol. 2019, 135, 221–235
  \item Sun, G., Hu, Z., {\bf Sun, F.~}, et al. (2017) An analysis on the influence of spatial scales on sensible heat fluxes in the north Tibetan Plateau based on Eddy covariance and large aperture scintillometer data. Theological and Applied Climatology. 129:965. \url{https://doi.org/10.1007/s00704-016-1809-7} 
  \item Gerken, T; Babel, W; Herzog, M; *Fuchs, K; {\bf Sun, F.}~; Ma, Y; Foken, T; Graf, HF, 2015, High-resolution modelling of interactions between soil moisture and convection development in mountain enclosed Tibetan basin, Hydrology and Earth System Sciences, 19, 4023-4040
  \item Li, M., W.~Babel, X.~Chen, L.~Zhang, {\bf F.~Sun}, B.~Wang, Y.~Ma, Z.~Hu, T.~Foken. 2015, A 3-year dataset of sensible and latent heat fluxes from the Tibetan Plateau, derived using eddy covariance measurements, Theoretical and Applied Climatology, 122:457–469, \url{https://doi.org/10.1007/s00704-014-1302-0}
  \item Gerken, T.~, W.~Babel, {\bf F.~Sun}, M.~Herzog, Y.~Ma, T.~Foken, and H.-F.~Graf, 2013, Uncertainty in atmospheric profiles and its impact on modeled convection development at NamCo Lake, Tibetan Plateau, Journal of Geophysical Research-Atmosphere, 118, 12317-12331, \url{https://doi.org/10.1002/2013JD020647}
  \item Chen, X.~, Z.~Su, Y.~Ma, and {\bf F.~Sun}, 2012, Analysis of land-atmosphere interactions over the north region of Mt. Qomolangma (Mt. Everest), Artic, Antarctic, and Alpine Research, 44(4), 412-422
  \item  Song, M., Ma Yaoming, Y. Zhang, M. Li, W. Ma, {\bf F. Sun}, 2011, Climate change features along the Brahmaputra Valley in the past 26 years and possible causes, Climatic Change, 106:649–660. \url{https://doi.org/10.1007/s10584-010-9950-2}
  \item Ma Yaoming, Y.~Wang, R.~Wu, Z.~Hu, K.~Yang, M.~Li, W.~Ma, L.~Zhong, {\bf F.~Sun}, X.~Chen, Z.~Zhu, S.~Wang, and H.~Ishikawa, 2009: Recent advances on the study of atmosphere-land interaction observations on the Tibetan Plateau, Hydrology and Earth System Sciences, 13, 1103-1111
  \item Li,~M., Y.~Ma, W.~Ma, Z.~Hu, H.~Ishikawa, Z.~Su, and {\bf F.~Sun}, 2006, Analysis of turbulence characteristics over the northern Tibetan Plateau area, Advances in Atmospheric Sciences, 23(4), 579-585
  \item 李茂善,阴蜀城,刘啸然,吕 钊,宋兴宇,马耀明,孙方林. 2019,近10年青藏高原及其周边湍流通量变化的数值模拟,高原气象,38(6),1140-1148
  \item  许洁, 马耀明*, 孙方林*, 马伟强, 2018,湖泊和上风向地形对纳木错地区秋季降水影响[J]. 高原气象, 37(6): 1535-1543
  \item 谷星月, 马耀明*, 马伟强*, 孙方林. 2018,青藏高原地表辐射通量的气候特征分析[J]. 高原气象, 37(6): 1458-1469
  \item 朱志鹍, 马耀明, 胡泽勇, 李茂善, 孙方林. 2015, 青藏高原那曲高寒草甸生态系统$CO_2$ 净交换及其影响因子[J]. 高原气象, 34(5):1217-1223
  \item 李茂善, 杨耀先, 马耀明, 孙方林, 陈学龙, 王宾宾, 朱志鹍. 2012, 纳木错(湖)地区湍流数据质量控制和湍流通量变化特征[J]. 高原气象, 31(4):875-884
  \item 宋敏红, 马耀明, 张宇, 李茂善, 马伟强, 孙方林. 2011, 雅鲁藏布江流域气温变化特征及趋势分析[J]. 气候与环境研究, 16(6):760-766
  \item 李茂善, 马耀明, 马伟强, Ishikawa Hirohiko, 孙方林, Ogino Shin-Ya. 2011. 藏北高原地区干,雨季大气边界层结构的不同特征[J]. 冰川冻土, 33(1): 72-79
  \item 田辉, 文军, 马耀明, 王介民, 吕世华, 张堂堂, 孙方林, 刘蓉. 2009, 夏季黑河流域蒸散发量卫星遥感估算研究[J]. 水科学进展, 20(1): 18-24
  \item  吕雅琼, 马耀明, 李茂善, 孙方林. 2008, 青藏高原纳木错湖区大气边界层结构分析[J]. 高原气象, 27(6):1205-1210
  \item 李茂善, 马耀明, 孙方林, 赵逸舟, 王永杰, 吕雅琼. 2008, 纳木错湖地区近地层微气象特征及地表通量交换分析[J]. 高原气象, 27(4):727-732
  \item 李茂善, 马耀明, 吕世华, 胡泽勇, Ishikawa Hirohiko, 马伟强, 孙方林, 宋敏红. 2008, 藏北高原地表能量和边界层结构的数值模拟[J]. 高原气象, 27(1):36-45
  \item 田辉, 马耀明, 文军, 李茂善, 孙方林, 马伟强. 2007, 秋季珠峰复杂地形下地表能量通量卫星遥感研究[J]. 高原气象, 26(6):1293-1299
  \item 陈学龙, 马耀明, 孙方林, 李茂善, 王树舟. 2007, 珠峰地区雨季对流层大气的特征分析[J]. 高原气象, 26(6):1280-1286
  \item 李茂善, 马耀明, Hirohiko Ishikawa, 马伟强, 孙方林, 王永杰, 朱志鲲2007, 珠穆朗玛峰北坡地区近地层及土壤微气象要素分析[J]. 高原气象, 26(6):1263-1268
  \item 孙方林, 马耀明. 2007, 珠穆朗玛峰北坡地区河谷局地环流特征观测分析[J]. 高原气象, 26(6):1187-1190
  \item 田辉, 马耀明, 胡晓, 陆登荣, 马伟强, 李茂善, 孙方林. 2007, 使用MODIS陆地产品LST和NDVI监测中国中、西部干旱[J]. 高原气象, 26(5):1086-1096
  \item 赵逸舟, 马耀明, 马伟强, 李茂善, 孙方林, 王磊, 向鸣. 2007, 藏北高原土壤温湿变化特征分析[J]. 冰川冻土, 29(4):578-583
  \item 孙方林, 马耀明, 马伟强, 李茂善. 2006, 珠峰地区大气边界层结构的一次观测研究[J]. 高原气象,
  25(6):1014-1019
  \item 马耀明, 仲雷, 田辉, 孙方林, 苏中波, Massimo Menenti. 2006, 青藏高原非均匀地表区域能量通量的研究[J]. 遥感学报, 10(4):542-547
  \item 马耀明, 马伟强, 李茂善, 孙方林, 王介民. 2004, 黑河中游非均匀地表能量通量的卫星遥感参数化[J]. 中国沙漠, 24(4):392-399
  \item 马耀明, 戴有学, 马伟强, 李茂善, 王介民, 文军, 孙方林. 2004, 干旱半干旱区非均匀地表区域能量通量的卫星遥感参数化[J]. 高原气象, 23(2):139-146
\end{etaremune}
\end{rSection}

\begin{rSection}{PRESENTATIONS AND POSTERS}
\begin{etaremune}
\item Sun, F., Y.~Ma, Z.~Hu, M.~Li, G.~Tartari, F.~Salerno, T.~Gerken, P.~Bonasoni, P.~Cristofanelli, and E.~Vuillermoz. Mechanism of Daytime Strong Winds on the Northern Slopes of Himalayas, near Mount Everest: Observation and Simulation. American Geophysical Union Fall Meeting, New Orleans, 11-15 December 2017. (POSTER)
\item Sun, F., Ma, Y; Hu, Z; Li, M. Observation of strong winds on northern slopes of Mount Everest. AOGS Anniversary Meeting, Brisbane, 24 - 28 June 2013. (ORAL)
\end{etaremune}
\end{rSection}
%----------------------------------------------------------------------------------------
%	TECHNICAL STRENGTHS SECTION
%----------------------------------------------------------------------------------------

\begin{rSection}{Technical Strengths}

\begin{tabular}{ @{} >{\bfseries}l @{\hspace{6ex}} l }
Languages & English(fluent), Chinese(native) \\
Computer Languages & C, Fortran, Python, NCL \\
Tools & TK3, EddyPro, WRF, RAMS, QGIS, GIMP\\
\end{tabular}

\end{rSection}

\begin{rSection}{REFERENCES}
Prof. Yaoming Ma (PhD Supervisor)\\Institute of Tibetan Plateau Research, Chinese Academy of Sciences\\
Email: ymma@itpcas.ac.cn; Phone: +86 10 6284 9698
\end{rSection}
%----------------------------------------------------------------------------------------
%	EXAMPLE SECTION
%----------------------------------------------------------------------------------------

%\begin{rSection}{Participation in Field Campaigns}
%\begin{rSubsection}{Heihe Integrated Remote Sensing Joint Experiment}{2007 - 2008}{Technical and Scientific Advisor}{Northwest China}
%\item Responsible for the installation and operation of GPS radio sounding system.
%\item Support for installation of Eddy Covariance System and Automatic weather station
%\end{rSubsection}
%%Section content\ldots

%\end{rSection}

%----------------------------------------------------------------------------------------

\end{document}
